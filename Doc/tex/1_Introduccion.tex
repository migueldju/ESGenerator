\capitulo{1}{Introducción}

El objetivo del trabajo es, como se explica en el resumen, desarrollar una aplicación que ayude a las empresas a \textbf{desarrollar de forma más rápida y sencilla sus informes ESG}, de acuerdo con la normativa de la Unión Europea.

ESG proviene de \textbf{Environmental, Social and Governance (Medioambiental, social y gubernamental)}, de modo que las empresas publican estas memorias para comunicar cuán grande ha sido su impacto en estos aspectos. 

Los 3 factores a estudiar se podrían dividir de la siguiente manera:

\begin{itemize}
    \item \textbf{Environmental:} asuntos relacionados con el impacto medioambiental, como emisiones de carbono, contaminación o consumo de recursos.
    \item \textbf{Social:} gestión de las relaciones humanas, prestando atención no solo a los empleados de la propia empresa, si no también a trabajadores de la cadena de valor y comunidades y consumidores relacionados con la empresa.
    \item \textbf{Governance:} evaluación de la estructura empresarial, centrándose en asuntos de transparencia y ética.
\end{itemize}

Estas memorias se llevan realizando en algunas empresas, sobre todo, en las más grandes, desde hace varias décadas, principalmente desde la creación, por parte de la ONU, de la \href{https://www.globalreporting.org/}{\textbf{Global Reporting Initiative (GRI)}}, en 1997. Esta iniciativa dio lugar a los primeros estándares sobre informes ESG, y se han ido actualizando con el paso de los tiempos, especialmente en esta última década.

Estos estándares no son obligatorios, salvo en unos pocos países, y existen ciertas bolsas de valores que las exigen a las empresas que quieran cotizar en ellas, por lo que solo un pequeño porcentaje de empresas, principalmente multinacionales, ha desarrollado informes siguiendo este marco de reporte.  

Cada organismo político ha ido desarrollando sus propias normativas en cuanto a informes ESG, aunque, en general, todos los países y organizaciones están dando cada vez más importancia a estos reportes.

En el caso de Europa, en 2014 se introdujo la \href{https://eur-lex.europa.eu/eli/dir/2014/95/oj/eng}{\textbf{NFRD (Non-Financial Reporting Directive)}}, que ya obligaba a las empresas con más de 500 empleados, así como a las financieras y a las que cotizasen en bolsa, a realizar ESG. Esta normativa, que afectaba a unas 11.000 empresas, no iba ligada a ningún estándar, de modo que las empresas podían elegir en cuál de los distintos marcos (GRI u otros como el TCFD, obligatorio en el Reino Unido).

Además de no seguir ningún estándar concreto, no se realizaban auditorías que verificasen la calidad de estos informes, por lo que no era una normativa lo suficientemente seria y eficaz de cara a cumplir con los propósitos medioambientales de la UE.

Por ello, en diciembre de 2022, la Unión Europea decidió poner solución a esto y sustituyó la NFRD por la \href{https://eur-lex.europa.eu/legal-content/EN/TXT/?uri=CELEX:32022L2464}{\textbf{CSRD (Corporate Sustainability Reporting Directive)}}. 

Esta directiva ha supuesto un cambio radical en el entorno empresarial de la Unión Europea, ya que obliga a muchas más empresas a realizar memorias ESG, siguiendo unos estándares únicos y teniendo que ser reguladas y validadas en auditorías de acuerdo a dichos estándares.
estructura de la memoria y del resto de materiales entregados.
